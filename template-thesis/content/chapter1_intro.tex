% !TeX spellcheck = en_US
% !TeX encoding = UTF-8
\chapter{Introduction}
\label{aufbau}

\section{Motivation}
\label{hinweise:titelblatt}
Industry 4.0, or the Fourth Industrial Revolution, has significantly transformed the industry, 
especially in the manufacturing sector by introducing smart technologies such as the Internet of Things (IoT), cloud connectivity, big data, Human-robot collaboration etc. 
This revolution has seen drastic improvements and innovation 
with enhanced efficiency and productivity being the core principle and driving force of innovation in Industry 4.0. Human-robot collaboration, 
a key component of Industry 4.0, has played a huge role in this advancement 
by bringing humans closer together and facilitating more efficient and cooperative workflows.  


Looking ahead to the future of the emerging Industry 5.0, the focus shifts 
towards a more human-centric approach. Industry 5.0 aims to balance technological 
advancements with human needs and interests, emphasizing sustainable and resilient
industrial practices. It seeks to merge the technological efficiency of Industry 
4.0 with a greater emphasis on enhancing human well-being and personalizing the
production process. Industry 5.0 brings back the human workforce to the factory,
where humans and machines are paired to increase process efficiency by utilizing 
human brainpower and creativity through the integration of workflows with intelligent systems \parencite{hum1}[1]. This shows a significant shift from purely efficiency-driven operations to those that also prioritize human factors and environmental sustainability. 


Traditionally, robots have been separated from human 
workers due to safety concerns and other factors.
 However, advancements in Industry 4.0 have significantly increased human-robot collaboration, bringing them closer together to jointly accomplish tasks. This evolution has seen robots move from being isolated behind safety barriers to now working side-by-side with human counterparts, effectively leveraging their unique strengths combining human flexibility and decision-making skills with the precision and consistency of robots.  


While technological advancements aim to optimize production, the comfort and well-being of human workers have not always been prioritized. This thesis aims to delve into the human aspect of human-robot interaction, considering how proximity to robots might affect the operator's physiological state. It aims to explore how continuous interaction with robots affects human stress levels. Monitoring and accurately assessing stress levels in human-robot collaborative environments is crucial. Elevated stress can lead to fear and doubt, negatively impacting productivity and the effectiveness of human-robot collaboration. This can also hinder the full utilization of the advanced capabilities of collaborative robots. Identifying and addressing these stress factors is key to optimizing the human-robot collaboration for enhanced productivity and a harmonious working environment. 


\section{Aim of the Thesis}
The title page provides information about the topic of the thesis, the chair, date of submission and the name of the author in the corresponding entries of the template.

\section{Literature Review}
\label{hinweise:kurzfassung}

T
\section{Literature Review}
