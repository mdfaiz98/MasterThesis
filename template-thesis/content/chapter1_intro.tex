% !TeX spellcheck = en_US
% !TeX encoding = UTF-8
\chapter{Introduction}
\label{aufbau}

\section{Motivation \gls*{gptmo}} 

Industry 4.0, also known as the Fourth Industrial Revolution, has brought about significant transformations, particularly in the manufacturing sector. This revolution has been characterized by the introduction of intelligent technologies such as the Internet of Things (IoT), cloud connectivity, big data, and human-robot collaboration. These advancements have led to notable improvements and innovations, with the core principle and driving force of innovation in Industry 4.0 being the enhancement of efficiency and productivity. Human-robot collaboration, a key component of Industry 4.0, has played a massive role in this advancement by bringing humans closer together and facilitating more efficient and cooperative workflows. 

Traditionally, industrial robots like robot manipulators, autonomous mobile robots, and gantry models have been kept separate from human workers primarily due to safety concerns. These robots, with their large size, substantial weight, and high speed, pose potential hazards when in close proximity to humans. This traditional approach prioritized the physical separation of robots and humans in industrial settings. However, advancements in Industry 4.0 have significantly increased the use of collaborative robots (cobots), bringing them closer together to accomplish tasks jointly. This evolutionary progression has witnessed the transformation of robots from being secluded behind safety barriers to now operating side-by-side with their human counterparts, effectively capitalizing on their unique capabilities which combine human adaptability and decision-making skills with the precision and consistency offered by robots.  

Looking towards the future of the emerging Industry 5.0, the focus shifts towards a more human-centric approach \parencite{industry5}. Industry 5.0 aims to strike a balance between technological advancements and human needs and interests. The goal is to merge the technological efficiency of Industry 4.0 with a greater emphasis on enhancing human operator's well-being and satisfaction.
Industry 5.0 seeks to address the human challenges related to Industry 4.0 by prioritizing the well-being of workers and placing them at the center of the manufacturing process.\parencite{hum1}.
This shows a significant shift from purely efficiency-driven operations to those that also prioritize human factors and environmental sustainability. 


With a focus on the human centrism aspect of Industry 5.0, this thesis aims to delve into the human aspect of human-robot interaction, considering how proximity to robots might affect the operator's physiological state. It aims to investigate how continuous interaction with robots impacts the stress levels experienced by humans and emphasizes the importance of monitoring and accurately assessing stress levels in human-robot collaborative environments. \textcite{Saupp} have previously indicated that cobots have the ability to influence the mental states of human workers, as they are often perceived as social entities. The close proximity of humans to robots in the workplace can lead to heightened stress levels, mainly if the robot's movements appear to be potentially harmful \parencite{Lasota}. For instance, if a co-robot swiftly moves towards a worker or follows an unpredictable path, it may induce feelings of anxiety or fear due to the perceived risk of sustaining an injury. This, in turn, can negatively impact both productivity and the efficacy of human-robot collaboration. Furthermore, it can impede the complete utilization of the advanced capabilities offered by collaborative robots. As robots become more autonomous and capable, identifying and addressing these stress factors is critical to optimizing human-robot collaboration for enhanced productivity and making the working environment effective, efficient and safe. 


\section{Aim of the Thesis\gls*{gptmo}}

The primary objective of this thesis is to evaluate the impact of varying collision avoidance strategies on human stress levels within the context of human-robot interaction. This involves conducting a study to collect and analyze data, aiming to understand the varying stress levels concerning different robot collision avoidance strategies while considering various collaboration levels and robot control strategies. Subsequently, the data obtained is utilized to develop a predictive model capable of identifying and addressing sources of stress during human-robot collaboration.


Specifically, the primary objectives of the thesis are: 
\begin{itemize}
	\item \textbf{Assessment of Human Stress Levels }: Develop a holistic approach for evaluating stress levels in human-robot interactions during different collaboration levels and robot control strategies, combining both objective physiological measures and subjective experiences. Objectively, the study will employ various physiological indicators such as \gls{GSR}, \gls{EDA}, \gls{HR}, and body posture analysis. These indicators will provide quantifiable data on the body's physiological response to robot interactions. Subjectively, the study aims to incorporate personal feedback from participants gathered through questionnaires. This will offer insights into their personal feelings and perceptions regarding their interactions with robots. By blending these objective and subjective methods, the study aims to understand stress in human-robot interactions comprehensively.
    This entails designing an acquisition system that successfully takes data from several sensors at different frequencies and synchronizes it. Devices such as the Empatica E4 wristband are used for gathering data on \gls{GSR}, \gls{EDA}, and other parameters, as well as a motion capture system to record human posture and movement. A vital aspect would be to synchronize these many data streams, ensuring accurate and consistent assessment of human physiological states across different robot interaction scenarios. 
    An important aspect was designing and conducting a subject study to collect data on participants' physiological responses while doing different assembly tasks under different human-robot interaction scenarios. These scenarios included three distinct levels of robot collision avoidance strategies: No collision avoidance, dynamic collision avoidance, and predictive collision avoidance, as well as three different collaboration levels: separated workspace with the cobot, shared workspace, and shared workspace with direct collaboration. The aim is to gather comprehensive data to analyze the impact of these varying robot control strategies on human stress levels. 
    \item \textbf{Stress Prediction Model}: Developing a model for predicting and classifying stress levels during human-robot collaboration. This model trained on the dataset of human physiological responses collected from the subject study. Various preprocessing and feature engineering techniques are used to prepare the data for the model.
    Various machine learning models, such as \gls{KNN}, \gls{SVM}, and others, are evaluated to determine the best model for predicting stress levels. The ultimate goal of this model is to provide a reliable tool for real-time monitoring and assessment of stress levels during human-robot collaboration. Accurately identifying stress-inducing factors and patterns can contribute to the development of proactive interventions and collision avoidance strategies that can enhance the overall well-being and performance of individuals engaged in human-robot interaction scenarios.

  \end{itemize}

\section{Related Work}
This and this and this have all done a comprehensive review of various 
stress detection and assessment methods. This literature review from our part relies on the knowledge gained by these authors.



\begin{comment}
subsection*{Physiological indicators of mental stress}
ome studies focus on reviewing the current state of affairs related to human stress detection. For instance, a review on human stress detection using bio-signals is presented in (Giannakakis et al., 2019). However, a discussion about the psychological, physical, and behavioral measures of human stress is found lacking. Further, publicly available databases for human stress measurement were also not explored. In another study, objective, subjective, physical, and behavioral measures for stress detection, as well as publicly available data used for human stress, are discussed. Another application-specific human stress measurement survey focusing on driver stress level is presented in (Rastgoo et al., 2018). Physical and physiological measures o

Mental Stress Detection using Data from Wearable and Non-Wearable Sensors: A Review 5 human stress for driver stress detection are explored in detail. The limitation of this survey is that it only discusses a specific application i.e., driver stress level, and is not generic.

\subsection*{Robots and stress}
\end{comment}


\section{Structure of the Thesis}
  In Chapter 2, we delve into the fundamental concepts of stress and examine the relationship between stress and key physiological signals such as \gls{EDA}, \gls{HR} and \gls{GSR}. Chapter 3 describes the data collection process of the subject study that was conducted to collect data on participants' physiological responses. Chapter 4 describes the data analysis process of the subject study  to analyze the collected data and the different pre-processing techniques and feature selection used to build the model. Chapter 5 describes the results of the subject study and the different machine learning models that were evaluated to determine the best model for predicting stress levels. Chapter 6 concludes the thesis and outlines the future work and research directions.
  

  
 
 