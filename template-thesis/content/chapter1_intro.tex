% !TeX spellcheck = en_US
% !TeX encoding = UTF-8
\chapter{Introduction}
\label{aufbau}

\section{Motivation}
Industry 4.0, also known as the Fourth Industrial Revolution, has brought about significant transformations within the industry, particularly in the manufacturing sector. This revolution has been characterized by the introduction of intelligent technologies such as the Internet of Things (IoT), cloud connectivity, big data, and human-robot collaboration, among others. These advancements have led to notable improvements and innovations, with the core principle and driving force of innovation in Industry 4.0 being the enhancement of efficiency and productivity. Human-robot collaboration, a key component of Industry 4.0, has played a huge role in this advancement by bringing humans closer together and facilitating more efficient and cooperative workflows. 

Looking towards the future of the emerging Industry 5.0, the focus shifts towards a more human-centric approach. Industry 5.0 aims to strike a balance between technological advancements and human needs and interests, placing a strong emphasis on sustainable and resilient industrial practices. The goal is to merge the technological efficiency of Industry 4.0 with a greater emphasis on enhancing human well-being and personalizing the production process. Industry 5.0 brings back the human workforce to the factory, where humans and machines are paired to increase 
process efficiency by utilizing human brainpower and creativity through the integration 
of workflows with intelligent system \parencite{hum1}.  This shows a significant shift from purely efficiency-driven operations to those that also prioritize human factors and environmental sustainability. 

Traditionally, industrial robots like manipulator arms, autonomous mobile robots, and gantry models have been kept separate from human workers primarily due to concerns regarding safety. These robots are typically characterized by their large size, substantial weight, and high speed, attributes that pose potential hazards when in close proximity to humans. Consequently, their design is predominantly focused on fulfilling specific tasks such as drilling, welding, or loading and unloading where their size and speed are necessary for efficiency but also necessitate isolation from human workers to ensure safety. This traditional approach prioritized the physical separation of robots and humans in industrial settings. However, advancements in Industry 4.0 have significantly increased the use of collaborative robots (cobots) , bringing them closer together to jointly accomplish tasks. This evolutionary progression has witnessed the transformation of robots from being secluded behind safety barriers to now operating side-by-side with their human counterparts, effectively capitalizing on their unique capabilities which combine human adaptability and decision-making skills with the precision and consistency offered by robots. 


While technological advancements aim to optimize production, the comfort and well-being of human workers have not always been prioritized. This thesis aims to delve into the human aspect of human-robot interaction, considering how proximity to robots might affect the operator's physiological state. It aims to investigate how continuous interaction with robots impacts the stress levels experienced by humans and emphasizes the importance of monitoring and accurately assessing stress levels in human-robot collaborative environments. \textcite{Saupp} have previously indicated that co-bots have the ability to influence the mental states of human workers, as they are often perceived as social entities. The close proximity of humans to robots in the workplace can lead to heightened levels of mental stress, particularly if the movements of the robot appear to be potentially harmful \parencite{Lasota}. For instance, if a co-robot swiftly moves towards a worker or follows an unpredictable path, it may induce feelings of anxiety or fear due to the perceived risk of sustaining an injury. This, in turn, can negatively impact both productivity and the efficacy of human-robot collaboration. Furthermore, it can impede the complete utilization of the advanced capabilities offered by collaborative robots. Identifying and addressing these stress factors is key to optimizing the human-robot collaboration for enhanced productivity and make the working environment effective, efficient and safe. 


\section{Aim of the Thesis}
The aim of this thesis is to evaluate the impact of varying collision avoidance strategies on human stress levels in the context of human-robot interaction. Our objective was to conduct a study to collect and analyze data to understand the different stress levels in relation to varying robot collision avoidance strategies. This is done taking different collaboration levels and robot control strategies into account. We then used this data to create a predictive model that can identify and address sources of stress during human robot collaboration.

Specifically, the objectives of the thesis are: 
\begin{itemize}
	\item \textbf{Assessment of Human Stress Levels}: Develop a holistic approach for evaluating stress levels in human-robot interactions, combining both objective physiological measures and subjective experiences. Objectively, the study will employ various physiological indicators such as Galvanic Skin Response (GSR), Electrodermal Activity (EDA), Heart Rate(HR), and body posture analysis. These indicators will provide quantifiable data on the body's physiological response to robot interactions. Subjectively, the study aims to incorporate personal feedback from participants, gathered through questionnaires. This will offer insights into their personal feelings and perceptions regarding their interactions with robots. By blending these objective and subjective methods, the study aims to provide a comprehensive understanding of stress in human-robot interactions.
	\item \textbf{Development of Data Acquisition and Synchronization System:} Designing an acquisition system that successfully takes data from several sensors at different frequencies and synchronizes it. Devices such as the Empatica E4 wristband are utilized for gathering data on Galvanic Skin Response (GSR), Electrodermal Activity (EDA), and other parameters, as well as a motion capture system to record human posture and movement. A vital aspect would be to synchronize these many data streams, ensuring accurate and consistent assessment of human physiological states across different robot interaction scenarios. 
	\item \textbf{Data Collection and Evaluation}: Designing and conducting a subject study to collect data on participants' physiological responses while doing different assembly tasks under different robot-human interaction scenarios. These scenarios included three distinct levels of robot collision avoidance strategies: No Collision Avoidance, Dynamic Collision Avoidance, and Predictive Collision Avoidance as well as three different collaboration levels: Different Workspace with the cobot, Shared Workspace, and Shared Workspace with Direct Collaboration. The aim is to gather comprehensive data to analyze the impact of these varying robot control strategies on human stress levels. 
	\item \textbf{Stress Prediction Model}: Developing a model for predicting and classifying stress levels during human robot collaboration. This model  trained on the dataset of human physiological responses collected from the subject study.Various preprocessing techniques and feature engineering techniques are used to prepare the data for the model.
	Various machine learning models such as K-Nearest Neighbors (KNN), Support Vector Machines (SVM), and others, are evaluated to determine the best model for predicting stress levels.
  \end{itemize}


  \section{Structure of the thesis}
  Chapter 1 presents the motivation behind the study and the aims of the thesis. In Chapter 2, we delve into the fundamental concepts of stress and examine the relationship between stress and key physiological signals such as Electrodermal Activity(EDA), heart rate (HR) and Galvanic Skin Response (GSR)etc. Chapter 3 describes the data collection process of the subject study that was conducted to collect data on participants' physiological responses. Chapter 4 describes the data analysis process of the subject study  to analyze the collected data and the different pre-processing techniques and feature selection used to build the model. Chapter 5 describes the results of the subject study and the different machine learning models that were evaluated to determine the best model for predicting stress levels. Chapter 6 concludes the thesis and outlines the future work and research directions.
  

  
 
 