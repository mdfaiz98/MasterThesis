% !TeX encoding = UTF-8

\chapter*{\iftoggle{lang_eng}{Abstract}{Kurzfassung}}

The integration of collaborative robots into shared workspaces with humans has become a pivotal aspect of modern manufacturing processes, heralding a new era of human-robot interaction. This thesis addresses the human element in these interactions by investigating the impact of varying robot collision avoidance strategies on human stress levels. A series of subject study were conducted to collect physiological, motion capture, and subjective data, which were then used to analyze the perceived stress levels of participants engaged in tasks with collaborative robots across various scenarios. Different collision avoidance strategies were implemented to study their impact on human stress levels.
Another key aspect of the research was the development of a stress prediction model. This model, trained on the data collected thought the subject study, utilizing various preprocessing and feature engineering techniques. Machine learning models, including K-Nearest Neighbors, Support Vector Machines and Random Forest, among others were evaluated to identify the most effective approach for predicting stress levels. Notably, Random Forest and Support Vector Machines achieved the highest accuracy of 94\% and 91.6\% respectively.
In summary, this thesis elucidates the nuanced relationship between cobot interaction strategies and human stress, providing valuable insights for designing more intuitive and stress-reducing HRI environments. The findings underscore the potential of tailored machine learning solutions in enhancing the well-being and productivity of human workers in collaborative settings.
In summary, this thesis explores the relationship between the robot collision avoidance strategies to human stress levels, providing valuable insights for designing more intuitive and stress-reducing human robot collaborative environments.