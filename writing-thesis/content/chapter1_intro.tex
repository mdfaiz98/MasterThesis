% !TeX spellcheck = en_US
% !TeX encoding = UTF-8
\chapter{Introduction}
\label{aufbau}

The thesis should focus on the documentation of your own contribution and scientific results, in which an analysis, interpretation and evaluation of the applied methodology and the results are of key importance.

In general, scientific publications start with a study and review of related literature.
You should revisit the literature study throughout your thesis before engaging on a new
investigation or issue.
In addition to the TU Dortmund library, research literature on the internet from the following sites.

\emph{http://scholar.google.de/}

\emph{http://www.sciencedirect.com/}

\emph{http://citeseer.comp.nus.edu.sg/cs}

\emph{http://ieeexplore.ieee.org/search/advsearch.jsp}

\emph{http://www.springerlink.com/}

Evaluate the quality and relevance of publications in relation to the topic of your
thesis.
The purpose of the literature study is to obtain an overview, sound knowledge and awareness on established related approaches, methodologies and solutions. 
Your thesis should only contain references that are relevant to the thesis assignment.
The analysis of the literature and the practical requirements of the thesis provide 
guidance to formulate a concise and specific problem formulation.
The thesis should start with a scientific problem formulation

The problem formulation consists of just one or two sentences and should clarify the research problem, you aim to address and to why and where it is relevant. The problem formulation is constitutes the core of your thesis and provides the beacon if you lose track during your investigations and writing of thesis.

In developing the solution of the tasks and the presentation of the results utilize the methods and knowledge that you acquired during your studies in related courses.
Take care that the gathered data is described as objectively as possible and that your findings are supported by sufficient examinations and evidence.
The presentation should allow others to comprehend and reproduce your results.
The thesis should conclude with a discussion and interpretation of relevant findings.
The scope of a bachelor thesis is about 30 pages, the scope of a master thesis about 60 pages.
A more detailed guideline on how to structure and write a thesis and how to cite references and sources properly is provided \textcite{Leit1}.

\section{Motivation}
\label{hinweise:titelblatt}

\section{Aim of the Thesis}
The title page provides information about the topic of the thesis, the chair, date of submission and the name of the author in the corresponding entries of the template.

\section{Literature Review}
\label{hinweise:kurzfassung}

The summary (abstract) of about half a page should provide a brief outline on the motivation, problem and content of the thesis. The scope and the main result should become clear.It has to be clear what the work is about and what the main results are.


%\section{Table of Contents}
\label{hinweise:inhaltsverzeichnis}

The table of contents represents the logical structure of the thesis.
It helps to clarify the organization and framework of the thesis.
The level of detail should be chosen appropriately and should normally not contain more than two levels (section, subsection) of granularity per chapter.

%\section{Nomenclature}
\label{hinweise:nomenklatur}

The nomenclature includes the specification of all symbols, variables, abbreviations and their explanations throughout the thesis.
The nomencl package automatically generates the entries of the symbols and facilitates
managing the nomenclature.\\
\emph{http://www.ctan.org/tex-archive/macros/latex/contrib/nomencl/}.
The tex files are scanned by \textit{makenomenclature} after \textit{nomencl} is invoked.
The result is a file \textit{struktur.nlo} containing the entries.
The entries are processed with \textit{makeindex.exe} and then included with \textit{printnomenclature} into the main latex file.
For an example, see section~\ref{hinweise:gleichungen}.

\subsection*{How to set up the nomenclature compiler}

\subsubsection{TeXstudio}
\textit{Options} $\rightarrow$ \textit{Configure TeXstudio ...} $\rightarrow$ \textit{Commands} $\rightarrow$ line \textit{Makeindex}:
\begin{quotation}
makeindex.exe \%.nlo -s nomencl.ist -o \%.nls 
\end{quotation}

\noindent Test configuration: F11 or \textit{Tools} $\rightarrow$ \textit{Index}. \\
If successful, recreate PDF. \textit{Makeindex} must be reinvoked each time the nomenclature changes. \\
See \textit{nomenclature.tex} for examples of how to generate the nomenclature.

\subsubsection{Bugs}

If the spacings in the nomenclature are incorrect and thus the descriptions of the symbols are not displayed, it helps to set the indent manually.
To do this, in the file \textit{nomenclature.tex} extend the line with the command \textit{\textbackslash printnomenclature} to \textit{\textbackslash printnomenclature[<Einzug>]}.
\textit{<Einzug>} is the indentation size of the description.
A collection of $4\,\mathrm{cm}$. is similar to the default in this sample nomenclature (By default, the indent size is \textit{\textbackslash nomlabelwidth}.
For more information, see the \textit{nomencl} Pakets).

%\section{Thesis Structure and Organization}
%\label{hinweise:struktur}

It is difficult to provide specific guidelines about thesis content and structure.
Nevertheless, most scientific publications in engineering and
natural sciences share a common structure of presentation.
The proposed thesis organization may not apply in all cases, but is often a good
starting point to structure your thesis.
If in doubt, discuss the structure of your thesis with your supervisor.
As an example, the content of the written paper may be structured as follows:

\begin{itemize}
	\item introduction
	\item theoretical foundations
	\item your approach and investigations
	\item experimental/simulation results and analysis
	\item summary and outlook
\end{itemize}
